\documentclass{article}
\usepackage{graphicx}
\usepackage[utf8]{inputenc}
\usepackage[margin= 3.5cm, includefoot]{geometry}
\usepackage{caption}
\usepackage{subcaption}
\usepackage{amsmath}
\usepackage{mathtools}
\usepackage{float}
\usepackage{verbatim}
\usepackage{ulem}

\title{Mind Your Hinglish}
\author{\hspace{0.55cm}Aniket Gupta\hspace{2cm}Aayush Goyal\hspace{2cm}Anshuman Panda\\2019CS10327\hspace{2.2cm}2019CS10452\hspace{2.4cm}2019CS10463}

\date{October-November 2021}

\begin{document}

\maketitle

\section{Introduction}
Were you taught Hindi in your school? What about English? Have you ever thought ki inn dono mein se zyada frequently tum konsi language use krte ho? And did you just realize that the former line contains morphemes of both Hindi and English? Do you know this mixing of Hindi and English has got \sout{official} name called Hinglish? Most of the academic curicullums frown upon mixing of Hindi and English, so here we will be discussing more on the depths and breadths of Hinglish.
\\\\
Hinglish, the language used by most of the Indian natives nowadays is a blend of both Hindi and English. It is a macaronic use of English with Hindi morphemes involving code switching and code mixing between these languages where they are freely interchanged according to the ease of use. This is widely in use because of the ease of speaking and writing. Mostly the people are not very frequent in using English and nor are they very proficient in pure Hindi and hence are observed to use a mix of these both, that is, Hinglish. Such code mixing and code switching has crept it's way into advertisements, newspapers, magazines, TV shows, Bollywood movies as well as the corridors of political and corporate powers in India. 
% TODO: write some survey about the ease of use between these thre laanguages. 
\\\\
Have you ever thought about how did Hinglish came to be? Have you ever thought how it became the language of the common even if it is not taught in any school and curicullum? Even the poeple who have not received any formal education in any of these 2 languages are seen using it in their daily life. How long do you think Hinglish has been in existence? Is it limited to the modern Indian youth influenced by the social media platforms? or it has been in existence much before the social media platforms became popular? What all do you think has given rise to the rapid growth in popularity of hinglish? Is it limited to the "upper-class elite" Indians or is it used in the semi-urban and the rural centres as well? 

\section{Rise of Hinglish}
It is a general misconception that this language has came into existence because of the influence of modern social media platforms and internet on the modern Indian elite class groups. In fact the use of hinglish can be traced back to time when India was under british rule. The british government at that time used english but most of the Indian were not good in English. These interaction of english and hindi speakers led to the use of words of both the languages to some extent by both sides, the government and the people. The example of use of hinglish, during the british rule can be observed in the words of Ayodhya Prasad Khatri (1857-1905), a prominent hindi poet. He wrote:
\begin{quote}
    \centering
    \textit{Rent Law ka gham karen ya Bill of Income Tax ka?\\
    Kya karen apan nahiin hai sense right now-a-days.\\Darkness chhaaya hua hai Hind men chaaro taraf\\
    Naam ki bhi hai nahiin baaqi na light now-a-days.}
\end{quote} 


\section{Examples from Data}

\section{Problems of FIT}

\section{Conclusions}

\section{Notes and References}

\end{document}